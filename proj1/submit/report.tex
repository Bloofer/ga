\documentclass{article}

\usepackage{kotex}
\usepackage{amsmath,amssymb}
\usepackage{amsthm,lipsum}
\usepackage{graphicx}% http://ctan.org/pkg/graphicx
\usepackage{listings}
\usepackage{indentfirst}
\usepackage{makecell}
\usepackage{multirow}


%package for tree
\usepackage{tikz}
\usepackage{tikz-qtree}
\usetikzlibrary{trees}

\setlength{\hoffset}{-25pt}
\addtolength{\textwidth}{50pt}
\setlength{\voffset}{-45pt}
\addtolength{\textheight}{90pt}


\begin{document}
\title{ 유전 알고리즘 프로젝트 1 보고서 }
\author{ }
\date{\today}
\maketitle


{~~}


%%%%%%%%%%%%%%%%%%%%%%%%%%%%%%%%%%%%%%%%%%%%%%%%%%%%%%%%%%%%
\section{ 사용한 GA의 구조 }

이번 과제에서는 Maxcut 문제를 풀기 위하여 가장 기본적인 GA 연산자들만 적용하였다.
기본적으로 N개의 유전자 집합인 해를 Population을 랜덤으로 생성하고, 그것으로 GA을 실행한다.
기본적으로는 K개의 해를 Population으로부터 교체하는 Generation의 교체를 반복하는데, 그 과정은 유전자로부터 Crossover시킬 부모의 형질을 선택하는, Selection, 그리고 선택된 두개의 부모를 교차시키는 Crossover, 그리고 교차된 자식 유전자에 대해 특정 확률을 통해 변이시키는 Mutation. 이 과정을 통해 K개의 자손 해를 생성시키면, Population 집단으로부터 K개의 해와 교체시킨다. 이 과정이 하나의 Generation 교체 과정.

<Pseudo Code>

GA의 기본적인 실행 구조는 주어진 TimeConstraint 동안 여러 Generation을 거쳐서 최적의 해를 탐사하는 과정을 계속해서 거쳐나간다. 구현된 ga() 함수에서는 매 Generation, 즉 Select/Crossover/Mutation/Replace 과정을 TimeConstraint 안에서 계속해서 반복문을 돌면서 수행해나간다. 이러한 수행과정에서 계속해서 해집단인 Population은 진화해나가고, 

놈놈.......


%%%%%%%%%%%%%%%%%%%%%%%%%%%%%%%%%%%%%%%%%%%%%%%%%%%%%%%%%%%%
\section{ 해의 표현 }

그래프 인코딩은 기본적으로 인풋으로 주어진 그래프의 형태를 그대로 따라, (from, to, weight)를 가진 Edge의 집합으로 구성하였다. Unweighted 그래프의 경우 weight가 일정하게 주어지지만, 그대로 자료구조를 사용할 수 있고, Weighted 그래프의 경우 두말 할 필요없이 그대로 이 인코딩을 적용할 수 있기 때문이다.
해 - 유전자 인코딩, 
자료구조 설명
놈놈.......


%%%%%%%%%%%%%%%%%%%%%%%%%%%%%%%%%%%%%%%%%%%%%%%%%%%%%%%%%%%%
\section{ 사용한 연산자에 대한 설명 }

놈놈.......

%%%%%%%%%%%%%%%%%%%%%%%%%%%%%%%%%%%%%%%%
\subsection{실험 1}

가장 기본적으로 수행한 실험 결과. Population size n = 20, Replacement size k = 4를 사용하고, Selection 전략은 Random, Crossover는 1-point, mutation은 uniform 을 사용하였다.



%%%%%%%%exp table%%%%%%%%%
 \begin{table}[h]
 \begin{center}
\caption{실험 1 결과}
\begin{tabular}{cccc}
\hline\hline
케이스 & 평균 결과 & 최고 결과 & 표준편차\\
\hline\hline
$Unweighted 50$ & 95.9 & 96 & 0.2 \\
\hline
$Unweighted 100$ & 349.5 & 350 & 1.3\\
\hline
$Unweighted 500$ & 3070.3 & 3080 & 18.1\\
\hline
$Weighted 500$ & 4340.6 & 4360 & 36.3\\
\hline
$Weighted Chimera 297$ & 8035.4 & 8512 & 1084.4\\
\hline
\end{tabular}
\end{center}
\end{table}
%%%%%%%%exp table%%%%%%%%%
%%%%%%%%%%%%%%%%%%%%%%%%%%%%%%%%%%%%%%%%

%%%%%%%%%%%%%%%%%%%%%%%%%%%%%%%%%%%%%%%%
\subsection{실험 2}

가장 기본적으로 수행한 실험 결과. Population size n = 20, Replacement size k = 4를 사용하고, Selection 전략은 Random, Crossover는 1-point, mutation은 uniform 을 사용하였다.



%%%%%%%%exp table%%%%%%%%%
 \begin{table}[h]
 \begin{center}
\caption{실험 1 결과}
\begin{tabular}{cccc}
\hline\hline
케이스 & 평균 결과 & 최고 결과 & 표준편차\\
\hline\hline
$Unweighted 50$ & 95.9 & 96 & 0.2 \\
\hline
$Unweighted 100$ & 349.5 & 350 & 1.3\\
\hline
$Unweighted 500$ & 3070.3 & 3080 & 18.1\\
\hline
$Weighted 500$ & 4340.6 & 4360 & 36.3\\
\hline
$Weighted Chimera 297$ & 8035.4 & 8512 & 1084.4\\
\hline
\end{tabular}
\end{center}
\end{table}
%%%%%%%%exp table%%%%%%%%%
%%%%%%%%%%%%%%%%%%%%%%%%%%%%%%%%%%%%%%%%





%%%%%%%%%%%%%%%%%%%%%%%%%%%%%%%%%%%%%%%%%%%%%%%%%%%%%%%%%%%%
\section{ 결과 분석 }

놈놈.......

%%%%%%%%%%%%%%%%%%%%%%%%%%%%%%%%%%%%%%%%%%%%%%%%%%%%%%%%%%%%
\section{ 논의 }

놈놈.......

% if exists, reference
%\begin{thebibliography}{3}
%\bibitem{p200}
%Dawson Engler and Ken Ashcraft,
%`` RacerX: Effective, Static Detection of Race Conditions and Deadlocks'',
%\emph{SOSP’03}, 
%October 19–22, 2003
%\end{thebibliography}

%\appendix
%\section{Appendix Title}

%This is the text of the appendix, if you need one.

\end{document}

