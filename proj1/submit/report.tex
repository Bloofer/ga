\documentclass{article}

\usepackage{kotex}
\usepackage{amsmath,amssymb}
\usepackage{amsthm,lipsum}
\usepackage{graphicx}% http://ctan.org/pkg/graphicx
\usepackage{listings}
\usepackage{indentfirst}
\usepackage{makecell}
\usepackage{multirow}


%package for tree
\usepackage{tikz}
\usepackage{tikz-qtree}
\usetikzlibrary{trees}

\setlength{\hoffset}{-25pt}
\addtolength{\textwidth}{50pt}
\setlength{\voffset}{-45pt}
\addtolength{\textheight}{90pt}


\begin{document}
\title{ 유전 알고리즘 프로젝트 1 보고서 }
\author{ }
\date{\today}
\maketitle


{~~}


%%%%%%%%%%%%%%%%%%%%%%%%%%%%%%%%%%%%%%%%%%%%%%%%%%%%%%%%%%%%
\section{ 사용한 GA의 구조 }

이번 과제에서는 Maxcut 문제를 풀기 위하여 가장 기본적인 GA 연산자들만 적용하였다.
기본적으로 N개의 유전자 집합인 해를 Population을 랜덤으로 생성하고, 그것으로 GA을 실행한다.
기본적으로는 K개의 해를 Population으로부터 교체하는 Generation의 교체를 반복하는데, 그 과정은 유전자로부터 Crossover시킬 부모의 형질을 선택하는, Selection, 그리고 선택된 두개의 부모를 교차시키는 Crossover, 그리고 교차된 자식 유전자에 대해 특정 확률을 통해 변이시키는 Mutation. 이 과정을 통해 K개의 자손 해를 생성시키면, Population 집단으로부터 K개의 해와 교체시킨다. 이 과정이 하나의 Generation 교체 과정.


%%%%%%%%%%%%%%%%%%%%%%%%%%%%%%%%%%%%%%%%%%%%%%%%%%%%%%%%%%%%
\subsection{ 문제 인코딩 }

그래프 인코딩은 기본적으로 인풋으로 주어진 그래프의 형태를 그대로 따라, (from, to, weight)를 가진 Edge의 집합으로 구성하였다. Unweighted 그래프의 경우 weight가 일정하게 주어지지만, 그대로 자료구조를 사용할 수 있고, Weighted 그래프의 경우 두말 할 필요없이 그대로 이 인코딩을 적용할 수 있기 때문이다.
해 - 유전자 인코딩, 
자료구조 설명

놈놈.......


%%%%%%%%%%%%%%%%%%%%%%%%%%%%%%%%%%%%%%%%%%%%%%%%%%%%%%%%%%%%
\subsection{ GA의 세부 구조 }

<Pseudo Code>

GA의 기본적인 실행 구조는 주어진 TimeConstraint 동안 여러 Generation을 거쳐서 최적의 해를 탐사하는 과정을 계속해서 거쳐나간다. 구현된 ga() 함수에서는 매 Generation, 즉 Select/Crossover/Mutation/Replace 과정을 TimeConstraint 안에서 계속해서 반복문을 돌면서 수행해나간다. 이러한 수행과정에서 계속해서 해집단인 Population은 진화해나가고, 

놈놈.......

%%%%%%%%%%%%%%%%%%%%%%%%%%%%%%%%%%%%%%%%%%%%%%%%%%%%%%%%%%%%
\subsection{ 사용한 연산자에 대한 설명 }

Selection의 경우 처음에는 두개의 부모를 무작위로 선택하는 Random 방식을 사용하다가, Population의 각 해의 fitness를 계산하여 공간 탐색을 하는 Roulette Wheel 알고리즘을 사용하였다.

Crossover의 경우, 1-Point Crossover와 2-Point Crossover를 구현하여 실험해보았다. 1-Point Crossover와 2-Point Crossover 두가지를 사용하여 실험한 이유는, 단지 1-Point Crossover와 Multipoint Crossover의 차이를 비교해보기 위함이었는데, 부모 형질을 반영하지 않는 Random Point Crossover의 경우 둘 다 큰 차이는 없었다.

Mutation은 Uniform 방식을 사용하여 Chromosome내 Random한 지점들에 대해 각각 변이시키는 방식을 선택하였다. 처음에는 Mutation Probability를 1.0\%를 주고 실험하다가 Mutation Probability를 1.5\%를 주고 실험을 하는 식으로 진행하였다.

Replacement는 구해진 k개의 자손 들을 Population 중 가장 품질이 나쁜 k개의 해와 교체하는 Genitor-style 방식으로 구현하고 실험하였다.

놈놈.......

%%%%%%%%%%%%%%%%%%%%%%%%%%%%%%%%%%%%%%%%
\section{실험 및 결과}

실험 환경에 대한 설명. 
컴퓨터 환경 및 OS 환경 싱글 코어 및 싱글 스레드 프로그램에 대한 수행. 
실험 수행 시간 및 전체 수행 횟수.

놈놈.......

%%%%%%%%%%%%%%%%%%%%%%%%%%%%%%%%%%%%%%%%
\subsection{실험 1 n=100,k=20}

가장 기본적으로 수행한 실험 결과. Population size n = 100, Replacement size k = 20를 사용하고, Selection 전략은 Random, Crossover는 1-point, mutation은 uniform 을 사용하였다.

이번 과제에서는 Generation gap(Population size / Replacement size)을 0.2로 유지하면서, 실험을 진행하였다. 다른 여러가지 변수로 맞추어서 실험해본 결과 이 게 가장 품질이 좋았기 때문이다. 처음에는 단순하게 Population size가 크면, 초기에 탐색하는 공간에서 넓은 범위를 가질 확률이 높다고 생각해서 n을 크게 주고 실험해보았다. 다음 실험 결과에서 보이듯, n을 바꿔가면서 실험하였지만, 사실 n이 큰 것이 좋은 품질을 주지는 않았다. 그것보다 n의 크기를 적당하게 준 것이 오히려 좋은 결과로 나타날 수 있었다.



%%%%%%%%exp table%%%%%%%%%
 \begin{table}[h]
 \begin{center}
\caption{실험 1 결과}
\begin{tabular}{cccc}
\hline\hline
케이스 & 평균 결과 & 최고 결과 & 표준편차\\
\hline\hline
$Unweighted 50$ & 95.9 & 96 & 0.2 \\
\hline
$Unweighted 100$ & 349.5 & 350 & 1.3\\
\hline
$Unweighted 500$ & 3070.3 & 3080 & 18.1\\
\hline
$Weighted 500$ & 4340.6 & 4360 & 36.3\\
\hline
$Weighted Chimera 297$ & 8035.4 & 8512 & 1084.4\\
\hline
\end{tabular}
\end{center}
\end{table}
%%%%%%%%exp table%%%%%%%%%
%%%%%%%%%%%%%%%%%%%%%%%%%%%%%%%%%%%%%%%%


%%%%%%%%%%%%%%%%%%%%%%%%%%%%%%%%%%%%%%%%
\subsection{실험 2 n=50,k=10}

가장 기본적으로 수행한 실험 결과. Population size n = 50, Replacement size k = 10를 사용하고, Selection 전략은 Random, Crossover는 1-point, mutation은 uniform 을 사용하였다.

이 실험 결과도 실험 1과 비슷한 추이를 보였다. Population size는 유전자 해의 품질에 크게 영향을 미치지는 않는 것 같다. 하지만, 크기를 조금 낮추었을 때, 보이는 Generation evoltion 횟수의 증가가, 눈에 띄게 비교되어 다만 이것을 좀 더 optimal한 위치로 찾고자 계속해서 낮추어 실험해보기로 하였다.



%%%%%%%%exp table%%%%%%%%%
 \begin{table}[h]
 \begin{center}
\caption{실험 1 결과}
\begin{tabular}{ccccc}
\hline\hline
케이스 & 평균 결과 & 최고 결과 & 표준편차 & 진화횟수\\
\hline\hline
$Unweighted 50$ & 95.9 & 96 & 0.2 & 1\\
\hline
$Unweighted 100$ & 349.5 & 350 & 1.3 & 1\\
\hline
$Unweighted 500$ & 3070.3 & 3080 & 18.1 & 1\\
\hline
$Weighted 500$ & 4340.6 & 4360 & 36.3 & 1\\
\hline
$Weighted Chimera 297$ & 8035.4 & 8512 & 1084.4 & 1\\
\hline
\end{tabular}
\end{center}
\end{table}
%%%%%%%%exp table%%%%%%%%%
%%%%%%%%%%%%%%%%%%%%%%%%%%%%%%%%%%%%%%%%


%%%%%%%%%%%%%%%%%%%%%%%%%%%%%%%%%%%%%%%%
\subsection{실험 3 n=20,k=4 Generation gap 조절(최적)}

가장 기본적으로 수행한 실험 결과. Population size n = 20, Replacement size k = 4를 사용하고, Selection 전략은 Random, Crossover는 1-point, mutation은 uniform 을 사용하였다.

이 실험 결과에서는 실험 1,2의 진행에 맞추어 Population size를 계속해서 낮춘 것이었다. 사실 이 것 이외에도 중간에 실험으로 


%%%%%%%%exp table%%%%%%%%%
 \begin{table}[h]
 \begin{center}
\caption{실험 1 결과}
\begin{tabular}{cccc}
\hline\hline
케이스 & 평균 결과 & 최고 결과 & 표준편차\\
\hline\hline
$Unweighted 50$ & 95.9 & 96 & 0.2 \\
\hline
$Unweighted 100$ & 349.5 & 350 & 1.3\\
\hline
$Unweighted 500$ & 3070.3 & 3080 & 18.1\\
\hline
$Weighted 500$ & 4340.6 & 4360 & 36.3\\
\hline
$Weighted Chimera 297$ & 8035.4 & 8512 & 1084.4\\
\hline
\end{tabular}
\end{center}
\end{table}
%%%%%%%%exp table%%%%%%%%%
%%%%%%%%%%%%%%%%%%%%%%%%%%%%%%%%%%%%%%%%

%%%%%%%%%%%%%%%%%%%%%%%%%%%%%%%%%%%%%%%%
\subsection{실험 4 Mutation prob 조정 - 1\% (실패)}

가장 기본적으로 수행한 실험 결과. Population size n = 20, Replacement size k = 4를 사용하고, Selection 전략은 Random, Crossover는 1-point, mutation은 uniform 을 사용하였다.

놈놈.......


%%%%%%%%exp table%%%%%%%%%
 \begin{table}[h]
 \begin{center}
\caption{실험 1 결과}
\begin{tabular}{cccc}
\hline\hline
케이스 & 평균 결과 & 최고 결과 & 표준편차\\
\hline\hline
$Unweighted 50$ & 95.9 & 96 & 0.2 \\
\hline
$Unweighted 100$ & 349.5 & 350 & 1.3\\
\hline
$Unweighted 500$ & 3070.3 & 3080 & 18.1\\
\hline
$Weighted 500$ & 4340.6 & 4360 & 36.3\\
\hline
$Weighted Chimera 297$ & 8035.4 & 8512 & 1084.4\\
\hline
\end{tabular}
\end{center}
\end{table}
%%%%%%%%exp table%%%%%%%%%
%%%%%%%%%%%%%%%%%%%%%%%%%%%%%%%%%%%%%%%%

%%%%%%%%%%%%%%%%%%%%%%%%%%%%%%%%%%%%%%%%
\subsection{실험 5 Selection 전략의 변경 - Roulette Wheel(성공)}

가장 기본적으로 수행한 실험 결과. Population size n = 20, Replacement size k = 4를 사용하고, Selection 전략은 Random, Crossover는 1-point, mutation은 uniform 을 사용하였다.

놈놈.......


%%%%%%%%exp table%%%%%%%%%
 \begin{table}[h]
 \begin{center}
\caption{실험 1 결과}
\begin{tabular}{cccc}
\hline\hline
케이스 & 평균 결과 & 최고 결과 & 표준편차\\
\hline\hline
$Unweighted 50$ & 95.9 & 96 & 0.2 \\
\hline
$Unweighted 100$ & 349.5 & 350 & 1.3\\
\hline
$Unweighted 500$ & 3070.3 & 3080 & 18.1\\
\hline
$Weighted 500$ & 4340.6 & 4360 & 36.3\\
\hline
$Weighted Chimera 297$ & 8035.4 & 8512 & 1084.4\\
\hline
\end{tabular}
\end{center}
\end{table}
%%%%%%%%exp table%%%%%%%%%
%%%%%%%%%%%%%%%%%%%%%%%%%%%%%%%%%%%%%%%%

%%%%%%%%%%%%%%%%%%%%%%%%%%%%%%%%%%%%%%%%
\subsection{실험 6 Crossover 전략의 변경 - Multipoint Crossover(실패)}

가장 기본적으로 수행한 실험 결과. Population size n = 20, Replacement size k = 4를 사용하고, Selection 전략은 Random, Crossover는 1-point, mutation은 uniform 을 사용하였다.

놈놈.......


%%%%%%%%exp table%%%%%%%%%
 \begin{table}[h]
 \begin{center}
\caption{실험 1 결과}
\begin{tabular}{cccc}
\hline\hline
케이스 & 평균 결과 & 최고 결과 & 표준편차\\
\hline\hline
$Unweighted 50$ & 95.9 & 96 & 0.2 \\
\hline
$Unweighted 100$ & 349.5 & 350 & 1.3\\
\hline
$Unweighted 500$ & 3070.3 & 3080 & 18.1\\
\hline
$Weighted 500$ & 4340.6 & 4360 & 36.3\\
\hline
$Weighted Chimera 297$ & 8035.4 & 8512 & 1084.4\\
\hline
\end{tabular}
\end{center}
\end{table}
%%%%%%%%exp table%%%%%%%%%
%%%%%%%%%%%%%%%%%%%%%%%%%%%%%%%%%%%%%%%%

%%%%%%%%%%%%%%%%%%%%%%%%%%%%%%%%%%%%%%%%
\subsection{실험 7 g++ 최적화 옵션 사용 -O3}

가장 기본적으로 수행한 실험 결과. Population size n = 20, Replacement size k = 4를 사용하고, Selection 전략은 Random, Crossover는 1-point, mutation은 uniform 을 사용하였다.

놈놈.......


%%%%%%%%exp table%%%%%%%%%
 \begin{table}[h]
 \begin{center}
\caption{실험 1 결과}
\begin{tabular}{cccc}
\hline\hline
케이스 & 평균 결과 & 최고 결과 & 표준편차\\
\hline\hline
$Unweighted 50$ & 95.9 & 96 & 0.2 \\
\hline
$Unweighted 100$ & 349.5 & 350 & 1.3\\
\hline
$Unweighted 500$ & 3070.3 & 3080 & 18.1\\
\hline
$Weighted 500$ & 4340.6 & 4360 & 36.3\\
\hline
$Weighted Chimera 297$ & 8035.4 & 8512 & 1084.4\\
\hline
\end{tabular}
\end{center}
\end{table}
%%%%%%%%exp table%%%%%%%%%
%%%%%%%%%%%%%%%%%%%%%%%%%%%%%%%%%%%%%%%%

%%%%%%%%%%%%%%%%%%%%%%%%%%%%%%%%%%%%%%%%%%%%%%%%%%%%%%%%%%%%
\section{ 결과 분석 }

실험 결과 Population size와 Replacement size는 품질에 크게 영향을 주지 않는 다는 것을 확인할 수 있었다. 너무 급진적인 변수만 아니라면.

그 대신, 그 외에 여러 가지 Selection 전략이라던가, Crossover 전략을 바꾸었을 때 품질에 대한 영향을 주는 요소가 더 많았다.

전통적인 GA를 돌렸을 때의 한계점.

마지막으로 가장 최종적으로 우수한 해를 나타낸 n=20, k=4, MutationProb=0.015, Roulette Wheel Selection, One-point Crossover와 g++ -O3 옵션을 통해 실험한 결과에서 수행된 Population의 진화 양상을 분석하였다.
초기 해의 수렴. 그 후에....

놈놈.......

%%%%%%%%%%%%%%%%%%%%%%%%%%%%%%%%%%%%%%%%%%%%%%%%%%%%%%%%%%%%
\section{ 논의 }

사실 보고서에 안썼지만, 품질 변화가 눈에 띄게 보이는 품질 향상을 만들어내기는 어려웠다.

여러번의 시행착오를 겪어서 가장 최적의 우수한 해를 찾아내는 변수들을 어느 정도 찾아낼 수 있었다.

실험 관찰 결과 재미있었던 것은 초기 initial population 생성된 해의 품질이 결과적으로 나타난 해에 영향을 많이 미친다는 것이었다. 변수의 변화보다..

초기에 Population을 문제에 대한 정보를 이용하여 잘 줄 수 있으면 좀 더 좋은 해를 얻을 수 있을 것 같다.

가장 전통적인 GA 연산자들로만 사용하여 우수한 해를 찾아내는 것에 대해 어려움이 많았다.

최적해에 어느정도 도달하였지만 그래도 많이 부족한 품질을 보임.

최적화와 지역 최적화 등 여러가지 방법을 사용하면 더 좋은 품질을 낼 수 있을 것 같다.

놈놈.......

% if exists, reference
%\begin{thebibliography}{3}
%\bibitem{p200}
%Dawson Engler and Ken Ashcraft,
%`` RacerX: Effective, Static Detection of Race Conditions and Deadlocks'',
%\emph{SOSP’03}, 
%October 19–22, 2003
%\end{thebibliography}

%\appendix
%\section{Appendix Title}

%This is the text of the appendix, if you need one.

\end{document}

